\documentclass{scrartcl}

\usepackage[english]{babel}
\usepackage[utf8]{inputenc}
\usepackage[T1]{fontenc} 
\usepackage{lmodern}
\usepackage{csquotes}

\usepackage{hyperref}

\newcommand{\messtype}[1]{\emph{#1}}

\newcommand{\checkthis}[1]{#1}

\title{AaS Protocol – AaSP}
\author{Rebekka Hubert \and Michael Staniek \and Simon Will}
\date{December 6, 2016}

\begin{document}

\maketitle

\section[Introduction]{Introduction\footnote{This section heavily borrows from the Introduction to the specification of the HTTP as given in RFC~2616. \url{https://www.w3.org/Protocols/rfc2616/rfc2616-sec1.html}}}
\label{sec:Introduction}

\subsection{Purpose}
\label{sub:Purpose}

The Annotation as Search Protocol (AaSP) is an application-level protocol for communication of an AaS server and an AaS client.
The purpose of this communication is described in \checkthis{the AaS specification}. % Provide a link here.

\subsection{Protocol compliance}
\label{sub:Protocol compliance}

The key words \enquote{MUST}, \enquote{MUST NOT}, \enquote{REQUIRED}, \enquote{SHALL}, \enquote{SHALL NOT}, \enquote{SHOULD}, \enquote{SHOULD NOT}, \enquote{RECOMMENDED},  \enquote{MAY}, and \enquote{OPTIONAL} in this document are to be interpreted as described in RFC 2119.\footnote{\url{https://www.w3.org/Protocols/rfc2616/rfc2616-sec1.html}}

An implementation of the AaSP is called compliant if and only if it satisfies all of the requirements marked by the key words \enquote{MUST}, \enquote{SHALL} or \enquote{REQUIRED}.
Furthermore, implementations are encouraged to comply to the recommendations marked by the keywords \enquote{SHOULD} and \enquote{RECOMMENDED} if they are suitable to the respective application.

\subsection{Overall Operation}
\label{sub:Overall Operation}

The client initiates the AaSP communication by sending a message of the type \messtype{request} to the server.
The server then build a forest of annotated trees from the request, generates a question, and reponds to the client by sending a message of the type \messtype{question}.
The client \checkthis{answers the question as true or false} and sends an \messtype{answer} message to the server.

When receiving an \messtype{answer}, the server filters its forest to only retain the trees which satisfy the given answer.
If only one tree remains, the server sends the remaining tree to the client in a \messtype{solution} message.
Otherwise it sends a \messtype{question} message again.

At any time, the client may send an \messtype{abort} message to end the communication prematurely, or it may send an \messtype{undo} message to revoke one or more of the answers given previously.

If the server encounters an unexpected situation it may send an \messtype{error} message to the client.

\section{AaSP Messages}
\label{sec:AaSP Messages}

\end{document}
